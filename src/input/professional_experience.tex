\begin{cvcontentsection}{Professional Experience}

\begin{cvjob}
    {Industrial \& Systems Engineering and Petroleum Engineering, Texas A\&M University}
    {College Station, TX}
    {Research Assistant}
    {Sep 2023}{Present}
    \item Created machine learning methodology for long-term risk assessment in CCS with interpretable surrogates, achieving 500\( \times \) training speedup compared to CMG's GEM with prediction errors around 2\%.
\end{cvjob}

\begin{cvjob2}
    {CNPC USA}
    {Houston, TX}
    {Research Intern}
    {Jun 2025 \textendash\ Aug 2025}{Jun 2024 \textendash\ Aug 2024}
    \item Developed an optimization framework to match drill bit simulations to laboratory tests without repeated, expensive simulator runs, reducing weight-on-bit error from 68\% to 5\% on unseen data.
    \item Engineered and simulated bit profiles embodying key design decisions, facilitating insight into dynamics and whirl behavior in advanced drill bit configurations.
    \item Advanced development of a drilling dynamics research tool, enabling seamless analysis of simulator, laboratory and field sensor data.
    Streamlined installation, reducing user effort by more than 80\% and leveraged Pythonic solutions to eliminate repeated work.
\end{cvjob2}

\begin{cvjob}
    {Industrial, Manufacturing \& Systems Engineering, Texas Tech University}
    {Lubbock, TX}
    {Research Assistant}
    {Sep 2021}{Aug 2023}
    \item Engaged with General Motors to model and optimize assembly line throughput via statistical methods and machine learning.
    \item Constructed a large-scale, high-volume dataset by leveraging SQL and developing custom web scrapers, informed by insights gained through on-site visits, enabling robust and domain-relevant model training.
    \item Optimized code to satisfy runtime requirements, later integrated into production-grade pilots in assembly plants.
\end{cvjob}

\begin{cvjob}
    {Applied Computational Intelligence Laboratory, PUC-Rio}
    {Rio de Janeiro, Brazil}
    {Researcher}
    {Mar 2018}{Jul 2021}
    \item Conceived a GAN-LSTM coupled simulator surrogate in the context of reservoir optimization and value of information and flexibility brought by smart wells, reducing prediction error from 19\% to 10\% on the OLYMPUS benchmark.
    Project part of Petrobras grant no.\ ANP 19783--0.
    \item Spearheaded introduction of deep learning and explainable AI into industry-established workflows for subsea object detection.
    Project part of Petrobras grant no.\ ANP 21914--7.
    \item Managed end-to-end development of production-grade software to optimize subsea production systems, from initial design through deployment.
    Designed and implemented a tailored genetic algorithm coupled to various multiphase-flow simulators, complemented by an intuitive, user-friendly interface deployed on client premises.
    Project part of Petrobras grant no.\ ANP 21225--8.
\end{cvjob}

\begin{cvjob}
    {Applied Computational Intelligence Laboratory, PUC-Rio}
    {Rio de Janeiro, Brazil}
    {Intern}
    {May 2017}{Dec 2017}
    \item Carried out translation and formatting of technical papers in oil \& gas and AI domains.
\end{cvjob}

\begin{cvjob}
    {RBNA Consult}
    {Rio de Janeiro, Brazil}
    {Consultant}
    {Nov 2017}{}
    \item Participated in the valuation of two Petrobras rigs, implementing discounted cash flow analysis to determine appropriate sale values.
    Benchmarked against rig market research to ensure price accuracy, with final purchase prices aligning to indicated ranges.
\end{cvjob}

\begin{cvjob}*
    {National Agency of Petroleum, Natural Gas and Biofuels}
    {Rio de Janeiro, Brazil}
    {Intern}
    {Oct 2016}{Apr 2017}
    \item Analyzed oil \& gas fields' cash flow, development plans and related documents, with close contact to operator companies.
\end{cvjob}

\end{cvcontentsection}
