\cvcontentsection{Professional Experience}

\cvjob
    {Research Assistant}
    {Department of Industrial \& Systems Engineering, Texas A\&M University}
    {College Station, TX}
    {September 2023 to Present}

\begin{itemize}
    \item Research on the novel application of survival analysis methods towards building reservoir simulator
    surrogates.
    \item Created and successfully implemented a methodology that leverages the power of survival analysis to predict
    the risk of CO\textsubscript{2} leakage in long-term CCS projects.
    This greatly reduces the reliance in computationally expensive reservoir simulations while maintaining high
    interpretability and promoting model qualities such as sparsity and training cheapness.
    The methodology is the first instance of survival analysis applied to carbon storage optimization in the literature.
\end{itemize}

\cvjob
    {Research and Software Development Intern}
    {CNPC USA}
    {Houston, TX}
    {June 2024 to August 2024}

\begin{itemize}
    \item Actively contributed to the development of a research tool for the drilling dynamics group that processes and
    analyzes data coming from multiple varied sources, such as computer simulations, laboratory experiments and field
    sensors.
    \item Led the effort on implementing a deployment strategy for the software and successfully converted the tool to a
    \textit{pip install}-based installation, lowering the entry barrier for new users and completely removing the burden
    on the user to track dependencies.
    \item Created the corporation's GitHub organization and gave company-wide training sessions on how to use the
    instrument and migrate existing workflows to Git and GitHub.
    \item Supervised by Dr.\ Lance Endres.
\end{itemize}

\cvjob
    {Research Assistant}
    {Department of Industrial, Manufacturing \& Systems Engineering, Texas Tech University}
    {Lubbock, TX}
    {August 2021 to August 2023}

\begin{itemize}
    \item Part of a project in partnership with General Motors, which aimed to optimize the throughput of its assembly
    lines via statistical methods and machine learning.
    A pilot program was implemented in the General Motors Wentzville Assembly plant and two patents were published.
\end{itemize}

\cvjob
    {Researcher}
    {Applied Computational Intelligence Laboratory (ICA/PUC-Rio)}
    {Rio de Janeiro, Brazil}
    {March 2018 to July 2021}

\cvsubjob
    {DOREE}
    {Software that leverages genetic algorithms for the optimization of subsea production systems}

\begin{itemize}
    \item Part of the team tasked with building the program from the ground up.
    Custom-built a genetic algorithm to suit the problem's requirements, led the creation of front-end and back-end and
    managed their continuous development and maintenance.
    \item Coordinated the communication with the end client, taking point in understanding their needs and leading the
    project's decision-making process.
    \item Created libraries for communication with reservoir and flow simulation software, including parsing and writing
    models, as well as running simulations and interpreting results, in both local computers and HPC clusters.
\end{itemize}

\cvsubjob
    {Flexwell}
    {Methodology and software capable of estimating the value of information and flexibility brought by the use of smart
    wells}

\begin{itemize}
    \item Drove the search for gaps in the methodology open to improvement, conducting research and implementing new
    features.
    \item Research in the use of deep learning for reservoir simulator substitution.
    Successfully implemented a Long Short-Term Memory Network-based simulator proxy, leading to reduction in well
    control optimization time.
    \item Conceived and developed a methodology for the generation of reservoir data via Generative Adversarial
    Networks, diminishing the need for computationally expensive simulations in the training of the proxy.
\end{itemize}

\cvsubjob
    {MANNTIS}
    {Deep learning for subsea object detection and classification}

\begin{itemize}
    \item Research into various approaches and networks, aiming to obtain better results by tailoring the methodology to
    the data in question.
    \item Spearheaded the inclusion of the developed software in the end client's workflow, easing the introduction of
    deep learning to a traditionally human-based task.
\end{itemize}

\cvjob
    {Intern}
    {Applied Computational Intelligence Laboratory (ICA/PUC-Rio)}
    {Rio de Janeiro, Brazil}
    {May 2017 to February 2018}

\begin{itemize}
    \item Responsible for research, translation and formatting of technical papers, with further submission and
    acceptance to journals and periodicals in the Oil \& Gas and Artificial Intelligence areas.
\end{itemize}

\cvjob
    {Consultant}
    {RBNA Consult}
    {Rio de Janeiro, Brazil}
    {November 2017}

\begin{itemize}
    \item Part of the team responsible for the valuation of two Petrobras rigs, which were then led to bidding.
    \item Provided crucial insights to the project, such as in relation to tax and to the rig market.
    The discounted cash flow method was used in order to find a suitable sale value, which was then compared to research
    done on the rig market at the time as to ascertain an acceptable price range.
    \item The winning bid was within the range suggested to the client, as well as the final destination of the rigs.

\end{itemize}

\cvjob
    {Development and Production Intern}
    {Brazilian National Agency of Petroleum, Natural Gas and Biofuels}
    {Rio de Janeiro, Brazil}
    {October 2016 to April 2017}

\begin{itemize}
    \item Elaboration of technical papers to support the board of directors' decisions.
    \item Analysis of oil \& gas fields' cash flow, development plans and related documents, with close contact to the
    operator companies.
\end{itemize}
