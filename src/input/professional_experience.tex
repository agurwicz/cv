\begin{cvcontentsection}{Professional Experience}

\begin{cvjob}
    {Research Assistant}
    {Industrial \& Systems Engineering, Texas A\&M University}
    {College Station, TX}
    {September 2023 - Present}
    \item Conceived and authored work that led to \$103k grant from the \textit{Crisman Institute for Petroleum Research}.
    \item Created a framework for long-term risk assessment in CCS projects with interpretable AI surrogates and frugal reservoir simulations.
    \item Applied methodology to a proof-of-concept case study of long-term CO\textsubscript{2} storage in a saline aquifer.
    The machine learning models achieved a 500\( \times \) training speedup in relation to CMG's GEM while obtaining prediction errors around 2\%.
\end{cvjob}

\begin{cvjob}
    {Research Intern}
    {CNPC USA}
    {Houston, TX}
    {June 2025 - August 2025}
    \item Designed and simulated drill bits to investigate fundamental profile choices and their effects on bit-whirl behavior, improving drilling performance.
    \item Developed an optimization framework to align bit simulation results with laboratory data without repeated, expensive simulation runs.
\end{cvjob}

\begin{cvjob}
    {Research and Software Development Intern}
    {CNPC USA}
    {Houston, TX}
    {June 2024 - August 2024}
    \item Advanced development of a research tool for drilling dynamics that processes and facilitates visualization and analysis of data coming from sources such as computer simulations, laboratory experiments and field sensors.
    \item Converted tool to \textit{pip install}-based installation through use of GitHub Actions and deployed across company.
    \item Created corporation's GitHub organization and gave company-wide sessions on how to use the platform and migrate existing workflows.
\end{cvjob}

\begin{cvjob}
    {Research Assistant}
    {Industrial, Manufacturing \& Systems Engineering, Texas Tech University}
    {Lubbock, TX}
    {August 2021 - August 2023}
    \item Engaged with General Motors to model and optimize assembly line throughput via statistical methods and machine learning, leading to pilot program in the General Motors Wentzville Assembly plant.
\end{cvjob}

\begin{cvjob}
    {Researcher}
    {Applied Computational Intelligence Laboratory (ICA/PUC-Rio)}
    {Rio de Janeiro, Brazil}
    {March 2018 - July 2021}
    \item Built custom genetic algorithms and software to optimize subsea production systems.
    \item Created Python libraries for communication with reservoir and flow simulation software, including parsing and writing models, as well as running simulations and interpreting results, in both local computers and HPC clusters.
    \item Conceived a GAN-LSTM coupled simulator surrogate in the context of reservoir optimization and value of information and flexibility brought by smart wells.
    The novel coupling reduced prediction error from 18.93\% to 9.71\%.
    \item Spearheaded introduction of deep learning and explainable AI into established workflows for subsea object detection and classification.
\end{cvjob}

\begin{cvjob}
    {Intern}
    {Applied Computational Intelligence Laboratory (ICA/PUC-Rio)}
    {Rio de Janeiro, Brazil}
    {May 2017 - February 2018}
    \item Carried out research, translation and formatting of technical papers in oil \& gas and AI domains.
\end{cvjob}

\begin{cvjob}
    {Consultant}
    {RBNA Consult}
    {Rio de Janeiro, Brazil}
    {November 2017}
    \item Took part in valuation of two Petrobras rigs before being led to bidding.
    Implemented the discounted cash flow method to find a suitable sale value, then compared to research done on rig market to ascertain an acceptable price range.

\end{cvjob}

\begin{cvjob}*
    {Intern}
    {Brazilian National Agency of Petroleum, Natural Gas and Biofuels}
    {Rio de Janeiro, Brazil}
    {October 2016 - April 2017}
    \item Analyzed oil \& gas fields' cash flow, development plans and related documents, with close contact to operator companies.
\end{cvjob}

\end{cvcontentsection}
